\documentclass{beamer}


\usepackage{graphicx} % Allows including images
\usepackage{booktabs} % Allows the use of \toprule, \midrule and \bottomrule in tables

%----------------------------------------------------------------------------------------
%	TITLE PAGE
%----------------------------------------------------------------------------------------

\title[Short title]{Main Topic } % The short title appears at the bottom of every slide, the full title is only on the title page

\title[Your Short Title]{Pulsar Timing and Gravitational Waves}
\author{Zahra Kader}
%\institute{Where You're From}
\date{October 20th 2017}

\begin{document}
	
	\begin{frame}
	\titlepage
\end{frame}

% Uncomment these lines for an automatically generated outline.
%\begin{frame}{Outline}
%  \tableofcontents
%\end{frame}

\section{Introduction}

\begin{frame}{What is a pulsar?}

\begin{itemize}
	\item Pulsars are rapidly rotating neutron stars
	\item Supernova explosions of massive stars leave behind immensely dense and highly magnetized neutron stars
\end{itemize}
\begin{figure}
	\centering
	\includegraphics[width=6cm]{"../Desktop/Project/pulsar pic"}
	\caption{Pulsar shown with its rotation and magnetic axes misaligned}
	\label{fig:pulsar-pic}
\end{figure}
\end{frame}


\section{MSPs}

\begin{frame}{Millisecond pulsars}

\begin{itemize}
\item Millisecond pulsars are formed in binary systems
\item Pulsar in binary system is spun up due to accretion of matter from the companion star
\begin{figure}
	\centering
	\includegraphics[width=6cm]{"../Desktop/Project/accreting pulsar"}
	\caption{Accretion of pulsar from companion star}
	\label{fig:accreting-pulsar}
\end{figure}
\end{itemize}

\end{frame}

\begin{frame}{Hulse and Taylor}
\begin{itemize}
\item First to detect a pulsar in a binary system
\item General relativity states that gravitational radiation carries energy away from the system 
\item Orbits of the two masses should shorten with time
\end{itemize}

\begin{figure}
	\includegraphics[width=8cm]{C:/Users/Zahra/Desktop/Project/hulsetaylor}
	\caption{\label{fig:hulsetaylor.jpg}The Binary Pulsar}
\end{figure}

\end{frame}

\begin{frame}{Indirect gravitational wave detection}
\begin{itemize}
\item Measurements obtained for the advance of periastron and gravitational redshift were used to find masses 
\item The $\dot Pb$ expected from GR was ${(-2.403\pm 0.005)}\times 10^{-12}$
\item The measured value for $\dot Pb$ was ${(-2.3\pm 0.22)}\times 10^{-12}$
\end{itemize}
\end{frame}

\begin{frame}{Timing model for isolated pulsar\\Spin down frequency}
\begin{itemize}
\item The spin of the pulsar decreases as it ages
\item The equation for the spin down frequency is:
$$\phi(t)=\phi_0 +\omega t+\frac{1}{2}\dot{\omega} t^2$$ 
\item which has fitting functions:
$$
\psi_a(t)=
\begin{pmatrix}
1\\
t\\
t^2\\
\end{pmatrix}, a=1,2,3
$$
\end{itemize}
\end{frame}

\begin{frame}{Position errors, proper motion and parallax}

\begin{figure}
	\centering
	\includegraphics[width=0.7\linewidth]{../Desktop/Project/parallax}
	\caption{Pulsar observed from earth}
	\label{fig:parallax}
\end{figure}
Pulsar has latitudinal and longitudinal positions affected by:
\begin{itemize}
\item The circular rotation of the earth around the SSB
\item The precession of the earth due to the sun and moon  
\end{itemize}
\end{frame}


%\begin{equation}\label{error in latitude}
%\delta\alpha=\delta\alpha_{0}+\delta\mu_{\alpha}t
%\end{equation}
%\begin{equation}\label{error in longitude}
%\delta\lambda=\delta\lambda_{0}+\delta\mu_{\lambda}t
%\end{equation}
%\begin{equation}{\label{N with 8 fit functions}}
%\begin{aligned}
%r(t)=&\frac{\delta N_{0}}{\nu}+\frac{\delta \nu}{\nu}t+\frac{\delta \dot\nu }{2 \nu}t^2+\frac{a}{c \sqrt{2}}[\delta\alpha \sin{\alpha}(\sin{2 \pi t}-\cos{2 \pi t})\\
%&+\delta\lambda \cos{\alpha}(\sin{2 \pi t}+\cos{2 \pi t})] %+\frac{a}{4c}p\cos^{2}\alpha \sin{4 \pi t}
%\end{aligned}
%\end{equation}

\begin{frame}
The fitting functions due to the modelling of these effects are:
	$$
\psi_a=
\begin{pmatrix}
sin2 \pi t\\
cos2 \pi t\\
t cos2 \pi t\\
t sin2 \pi t\\
sin4 \pi t\\
\end{pmatrix}, a=4,5,6,7,8
$$

\begin{itemize}
	\item $\psi_4$ and $\psi_5$ account for position errors
	\item $\psi_6$ and $\psi_7$ account for proper motion errors
	\item $\psi_8$ accounts for errors in timing parallax
\end{itemize}
\end{frame} 

\begin{frame}{Timing residual plots}
\begin{itemize}
	\item Timing residual is the difference between expected and observed arrival times
	\item The observed data has white noise;random fluctuations in arrival time for each observation
\end{itemize}
\begin{figure}
	\centering
	\includegraphics[width=0.7\linewidth]{"../Desktop/Project/timing residual plots"}
	\caption{Characteristic plots for timing residual}
	\label{fig:timing-residual-plots}
\end{figure}
\end{frame}
\begin{frame}{Dispersion corrections}
\begin{itemize}
\item Particles in ISM cause dispersion of light which delays the arrival time
\item  Pulses emitted at lower radio frequencies travel slower than those emitted at higher frequencies
\begin{figure}
	\centering
	\includegraphics[width=5cm]{"../Desktop/Project/integrated pulse profile"}
	\caption{Pulse dispersion}
	\label{fig:integrated-pulse-profile}
\end{figure}
\end{itemize}
%$$T=t+\frac{D}{f^2}-\Delta R_s-\Delta E_s +\Delta S_s+\Delta R+\Delta E +\Delta S$$
\end{frame}

%\begin{frame}{Barycentric corrections}
%\begin{itemize}
%\item Spin down frequency equation requires time measured in an inertial reference %frame
%\end{itemize}
%$$T=t+\frac{D}{f^2}-\Delta R_s-\Delta E_s +\Delta S_s+\Delta R+\Delta E +\Delta S$$
%\end{frame}
\begin{frame}{Gravitational waves}
\begin{itemize}
	\item Waves of a quadrupole nature 
	\item Created from acceleration of non-spherically symmetric mass distributions 
\end{itemize}
\begin{figure}
	\centering
	\includegraphics[width=6cm]{"../Desktop/Project/GW polarisations"}
	\caption{Gravitational wave polarisations}
	\label{fig:gw-polarisations}
\end{figure}
\end{frame}


\begin{frame}{Gravitational wave detectors}
\begin{figure}
	\centering
	\includegraphics[width=0.7\linewidth]{../Desktop/Project/The_Gravitational_wave_spectrum_Sources_and_Detectors}
	\caption{Frequency range of gravitational wave detectors}
	\label{fig:thegravitationalwavespectrumsourcesanddetectors}
\end{figure}
\end{frame}

\begin{frame}{Gravitational wave sources\\Supermassive black hole binaries}
Most likely source of gravitational waves that can be detected using pulsar timing
\begin{figure}
	\centering
	\includegraphics[width=0.7\linewidth]{"../Desktop/Project/smbhb image"}
	\caption{Gravitational waves produced by coalescing masses}
	\label{fig:smbhb-image}
\end{figure}
\end{frame}

\begin{frame}{Primordial gravitational waves}
\begin{itemize}
	\item Originate from quantum fluctuations in the early universe 
	\item Their detection may prove the theory of inflation
\end{itemize}
\begin{figure}
	\centering
	\includegraphics[width=0.7\linewidth]{"../Desktop/Project/prim gws inflation"}
	\caption{Primordial gravitational waves}
	\label{fig:prim-gws-inflation}
\end{figure}
\end{frame}

\begin{frame}{Mean residual squared}
\begin{itemize}
	\item The residual is expressed as:
\begin{equation}\label{residual with beta}
r(t,\theta^*)=\epsilon(t)-\sum_{a=1}^{m}\beta_a \psi_a(t,\theta^*)%+O(\beta_a^2)
\end{equation}
\item which is used to obtain the expression for mean residual squared
\begin{equation}
<r^{2}(t)>=\int_{0}^{\infty}P(f)T(f)df
\end{equation}
where P(f) is the power spectrum and $$T(f)=1-\frac{1}{N}\sum_{a=1}^{8}\psi_a^{'}(f)\psi_a^{'*}(f)$$ is the transfer function
\end{itemize}
\end{frame}

\begin{frame}{Fitting functions}

\begin{itemize}
	\item Plot of the fitting functions that account for the decrease in spin frequency and timing parallax of the pulsar
\begin{figure}
	\centering
	\includegraphics[width=8cm]{"../Desktop/Project/blandford psi1238"}
	\caption{Fitting functions for $\psi_a(f), a=1,2,3,8$}
	\label{fig:blandford-psi1238}
\end{figure}
\end{itemize}
\end{frame}

\begin{frame}
\item Plot of the fitting functions that account for the errors in position and proper motion of the pulsar

\begin{figure}
	\centering
	\includegraphics[width=0.7\linewidth]{"../Desktop/Project/blandford psi4567"}
	\caption{Fitting functions for $|\psi_a(f)|^2, a=4,5,6,7$}
	\label{fig:blandford-psi4567}
\end{figure}
\end{frame}

\begin{frame} 
$$\LARGE \text{Thank you for listening}$$
\end{frame}
\end{document}

